\documentclass[12pt]{report}\usepackage[]{graphicx}\usepackage[dvipsnames]{xcolor}
% maxwidth is the original width if it is less than linewidth
% otherwise use linewidth (to make sure the graphics do not exceed the margin)
\makeatletter
\def\maxwidth{ %
  \ifdim\Gin@nat@width>\linewidth
    \linewidth
  \else
    \Gin@nat@width
  \fi
}
\makeatother

\definecolor{fgcolor}{rgb}{0.345, 0.345, 0.345}
\newcommand{\hlnum}[1]{\textcolor[rgb]{0.686,0.059,0.569}{#1}}%
\newcommand{\hlstr}[1]{\textcolor[rgb]{0.192,0.494,0.8}{#1}}%
\newcommand{\hlcom}[1]{\textcolor[rgb]{0.678,0.584,0.686}{\textit{#1}}}%
\newcommand{\hlopt}[1]{\textcolor[rgb]{0,0,0}{#1}}%
\newcommand{\hlstd}[1]{\textcolor[rgb]{0.345,0.345,0.345}{#1}}%
\newcommand{\hlkwa}[1]{\textcolor[rgb]{0.161,0.373,0.58}{\textbf{#1}}}%
\newcommand{\hlkwb}[1]{\textcolor[rgb]{0.69,0.353,0.396}{#1}}%
\newcommand{\hlkwc}[1]{\textcolor[rgb]{0.333,0.667,0.333}{#1}}%
\newcommand{\hlkwd}[1]{\textcolor[rgb]{0.737,0.353,0.396}{\textbf{#1}}}%
\let\hlipl\hlkwb

\usepackage{framed}
\makeatletter
\newenvironment{kframe}{%
 \def\at@end@of@kframe{}%
 \ifinner\ifhmode%
  \def\at@end@of@kframe{\end{minipage}}%
  \begin{minipage}{\columnwidth}%
 \fi\fi%
 \def\FrameCommand##1{\hskip\@totalleftmargin \hskip-\fboxsep
 \colorbox{shadecolor}{##1}\hskip-\fboxsep
     % There is no \\@totalrightmargin, so:
     \hskip-\linewidth \hskip-\@totalleftmargin \hskip\columnwidth}%
 \MakeFramed {\advance\hsize-\width
   \@totalleftmargin\z@ \linewidth\hsize
   \@setminipage}}%
 {\par\unskip\endMakeFramed%
 \at@end@of@kframe}
\makeatother

\definecolor{shadecolor}{rgb}{.97, .97, .97}
\definecolor{messagecolor}{rgb}{0, 0, 0}
\definecolor{warningcolor}{rgb}{1, 0, 1}
\definecolor{errorcolor}{rgb}{1, 0, 0}
\newenvironment{knitrout}{}{} % an empty environment to be redefined in TeX

\usepackage{alltt}

\usepackage[utf8]{inputenc}
\usepackage[spanish]{babel}
\usepackage[margin=2.54cm]{geometry}
\usepackage[dvipsnames]{xcolor}
\usepackage{array, amssymb, amsthm, enumitem, fancyhdr, float, graphicx, hyperref, hologo, mathtools, tikz, tikz-cd}
\usepackage[spanish, noabbrev]{cleveref}

\pagestyle{fancy}
\lhead{\footnotesize \leftmark}
\rhead{\footnotesize \rightmark}

\title{
	\huge
	\noindent\textbf{Fundamentos de la Ciencia de Datos}\\
	
	{\Large \textit{Práctica 1}}
	\vspace{1cm}
	
	\huge
	Grado en Ingeniería Informática\\
	Universidad de Alcalá\\
	
	\vspace{1cm}
	
	\includegraphics[scale=0.075]{img/logo}
}

\author{
	Pablo García García\\
	Abel López Martínez\\
	Álvaro Jesús Martínez Parra\\
	Raúl Moratilla Núñez
}

\date{
	\large{14 de noviembre de 2023}
}

\hypersetup{
	pdftitle={Práctica 1}, 
	pdfauthor={Pablo García García, Abel López Martínez, Álvaro Jesús Martínez Parra, Raúl Moratilla Núñez}, 
	pdfsubject={Fundamentos de la Ciencia de Datos}, 
	pdfcenterwindow, 
	pdfnewwindow=true, 
	pdfkeywords={Entrega de la PL1 de laboratorio correspondiente al Curso 2023-2024}, 
	bookmarksopen=true 
}
\IfFileExists{upquote.sty}{\usepackage{upquote}}{}
\begin{document}
	
\begin{knitrout}
\definecolor{shadecolor}{rgb}{0.969, 0.969, 0.969}\color{fgcolor}\begin{kframe}
\begin{alltt}
\hlstd{fichero} \hlkwb{=} \hlkwd{read.csv}\hlstd{(}\hlstr{"distancia_universitarios.csv"}\hlstd{)}
\hlstd{fichero}
\end{alltt}
\begin{verbatim}
##    Distancia
## 1       16.5
## 2       34.8
## 3       20.7
## 4        6.2
## 5        4.4
## 6        3.4
## 7       24.0
## 8       24.0
## 9       32.0
## 10      30.0
## 11      33.0
## 12      27.0
## 13      15.0
## 14       9.4
## 15       2.1
## 16      34.0
## 17      24.0
## 18      12.0
## 19       4.4
## 20      28.0
## 21      31.4
## 22      21.6
## 23       3.1
## 24       4.5
## 25       5.1
## 26       4.0
## 27       3.2
## 28      25.0
## 29       4.5
## 30      20.0
## 31      34.0
## 32      12.0
## 33      12.0
## 34      12.0
## 35      12.0
## 36       5.0
## 37      19.0
## 38      30.0
## 39       5.5
## 40      38.0
## 41      25.0
## 42       3.7
## 43       9.0
## 44      30.0
## 45      13.0
## 46      30.0
## 47      30.0
## 48      26.0
## 49      30.0
## 50      30.0
## 51       1.0
## 52      26.0
## 53      22.0
## 54      10.0
## 55       9.7
## 56      11.0
## 57      24.1
## 58      33.0
## 59      17.2
## 60      27.0
## 61      24.0
## 62      27.0
## 63      21.0
## 64      28.0
## 65      30.0
## 66       4.0
## 67      46.0
## 68      29.0
## 69       3.7
## 70       2.7
## 71       8.1
## 72      19.0
## 73      16.0
\end{verbatim}
\begin{alltt}
\hlstd{len} \hlkwb{=} \hlkwa{function}\hlstd{(}\hlkwc{list}\hlstd{)\{}
        \hlstd{count} \hlkwb{=} \hlnum{0}
        \hlkwa{for} \hlstd{(element} \hlkwa{in} \hlstd{list)\{}
                \hlstd{count} \hlkwb{=} \hlstd{count} \hlopt{+} \hlnum{1}
        \hlstd{\}}
        \hlstd{count}
\hlstd{\}}

\hlstd{distancias} \hlkwb{=} \hlstd{fichero}\hlopt{$}\hlstd{Distancia}

\hlstd{longitud} \hlkwb{=} \hlkwd{len}\hlstd{(distancias)}
\hlstd{longitud}
\end{alltt}
\begin{verbatim}
## [1] 73
\end{verbatim}
\begin{alltt}
\hlstd{bubble} \hlkwb{=} \hlkwa{function}\hlstd{(}\hlkwc{list}\hlstd{,} \hlkwc{asc} \hlstd{=} \hlnum{TRUE}\hlstd{)\{}
        \hlstd{n} \hlkwb{=} \hlkwd{len}\hlstd{(list)}
        \hlkwa{if}\hlstd{(asc)\{}
                \hlkwa{for} \hlstd{(i} \hlkwa{in} \hlnum{2}\hlopt{:}\hlstd{n)\{}
                        \hlkwa{for} \hlstd{(j} \hlkwa{in} \hlnum{1}\hlopt{:}\hlstd{(n}\hlopt{-}\hlnum{1}\hlstd{))\{}
                                \hlkwa{if} \hlstd{(list[j]} \hlopt{>} \hlstd{list[j}\hlopt{+}\hlnum{1}\hlstd{])\{}
                                        \hlstd{temp} \hlkwb{=} \hlstd{list[j]}
                                        \hlstd{list[j]} \hlkwb{=} \hlstd{list[j}\hlopt{+}\hlnum{1}\hlstd{]}
                                        \hlstd{list[j}\hlopt{+}\hlnum{1}\hlstd{]} \hlkwb{=} \hlstd{temp}
                                \hlstd{\}}
                        \hlstd{\}}
                \hlstd{\}}
        \hlstd{\}}
        \hlkwa{else} \hlstd{\{}
                \hlkwa{for} \hlstd{(i} \hlkwa{in} \hlnum{2}\hlopt{:}\hlstd{n)\{}
                        \hlkwa{for} \hlstd{(j} \hlkwa{in} \hlnum{1}\hlopt{:}\hlstd{(n}\hlopt{-}\hlnum{1}\hlstd{))\{}
                                \hlkwa{if} \hlstd{(list[j]} \hlopt{<} \hlstd{list[j}\hlopt{+}\hlnum{1}\hlstd{])\{}
                                        \hlstd{temp} \hlkwb{=} \hlstd{list[j]}
                                        \hlstd{list[j]} \hlkwb{=} \hlstd{list[j}\hlopt{+}\hlnum{1}\hlstd{]}
                                        \hlstd{list[j}\hlopt{+}\hlnum{1}\hlstd{]} \hlkwb{=} \hlstd{temp}
                                \hlstd{\}}
                        \hlstd{\}}
                \hlstd{\}}
        \hlstd{\}}
        \hlstd{list}
\hlstd{\}}
\hlstd{distanciasordenadas} \hlkwb{=} \hlkwd{bubble}\hlstd{(distancias,} \hlnum{FALSE}\hlstd{)}
\hlstd{distanciasordenadas}
\end{alltt}
\begin{verbatim}
##  [1] 46.0 38.0 34.8 34.0 34.0 33.0 33.0 32.0 31.4 30.0 30.0 30.0 30.0 30.0 30.0
## [16] 30.0 30.0 29.0 28.0 28.0 27.0 27.0 27.0 26.0 26.0 25.0 25.0 24.1 24.0 24.0
## [31] 24.0 24.0 22.0 21.6 21.0 20.7 20.0 19.0 19.0 17.2 16.5 16.0 15.0 13.0 12.0
## [46] 12.0 12.0 12.0 12.0 11.0 10.0  9.7  9.4  9.0  8.1  6.2  5.5  5.1  5.0  4.5
## [61]  4.5  4.4  4.4  4.0  4.0  3.7  3.7  3.4  3.2  3.1  2.7  2.1  1.0
\end{verbatim}
\begin{alltt}
\hlstd{rank} \hlkwb{=} \hlkwa{function}\hlstd{(}\hlkwc{list}\hlstd{)\{}
        \hlstd{ordered_list} \hlkwb{=} \hlkwd{bubble}\hlstd{(list)}
        \hlstd{ordered_list[}\hlkwd{len}\hlstd{(ordered_list)]} \hlopt{-} \hlstd{ordered_list[}\hlnum{1}\hlstd{]}
\hlstd{\}}

\hlstd{rango} \hlkwb{=} \hlkwd{rank}\hlstd{(distanciasordenadas)}
\hlstd{rango}
\end{alltt}
\begin{verbatim}
## [1] 45
\end{verbatim}
\begin{alltt}
\hlstd{absolute_freq} \hlkwb{=} \hlkwa{function}\hlstd{(}\hlkwc{list}\hlstd{)\{}
        \hlstd{ordered_list} \hlkwb{=} \hlkwd{bubble}\hlstd{(list)}
        \hlstd{n} \hlkwb{=} \hlkwd{len}\hlstd{(ordered_list)}
        \hlstd{elements} \hlkwb{=} \hlkwd{vector}\hlstd{()}
        \hlstd{frequencies} \hlkwb{=} \hlkwd{vector}\hlstd{()}
        \hlstd{i} \hlkwb{=} \hlnum{1}
        \hlkwa{while} \hlstd{(i} \hlopt{<=} \hlstd{n)\{}
                \hlstd{actual_element} \hlkwb{=} \hlstd{ordered_list[i]}
                \hlstd{elements} \hlkwb{=} \hlkwd{append}\hlstd{(elements, actual_element)}
                \hlstd{actual_freq} \hlkwb{=} \hlnum{0}
                \hlstd{j} \hlkwb{=} \hlstd{i}
                \hlkwa{while}\hlstd{(j} \hlopt{<=} \hlstd{n} \hlopt{&} \hlstd{actual_element} \hlopt{==} \hlstd{ordered_list[j])\{}
                        \hlstd{actual_freq} \hlkwb{=} \hlstd{actual_freq} \hlopt{+} \hlnum{1}
                        \hlstd{j} \hlkwb{=} \hlstd{j}\hlopt{+}\hlnum{1}
                \hlstd{\}}
                \hlstd{frequencies} \hlkwb{=} \hlkwd{append}\hlstd{(frequencies, actual_freq)}
                \hlstd{i} \hlkwb{=} \hlstd{j}
        \hlstd{\}}
        \hlkwd{rbind}\hlstd{(elements, frequencies)}
\hlstd{\}}
\end{alltt}
\end{kframe}
\end{knitrout}

PARTE 2









\begin{knitrout}
\definecolor{shadecolor}{rgb}{0.969, 0.969, 0.969}\color{fgcolor}\begin{kframe}
\begin{alltt}
union = \hlkwd{function}(c1, c2)\{
	\hlkwd{if} (\hlkwd{len}(c1) == 0)\{
		c2
	\}
	else \hlkwd{if} (\hlkwd{is.element}(c1[1], c2))\{
		\hlkwd{union}(c1[-1], c2)
	\}
	else\{
		\hlkwd{union}(c1[-1], \hlkwd{append}(c2, c1[1]))
	\}
\}

unionp = \hlkwd{union}(\hlkwd{c}(\hlstr{"P"},\hlstr{"A"}, \hlstr{"L"}), \hlkwd{c}(\hlstr{"P"},\hlstr{"A"}, \hlstr{"C"}, \hlstr{"N"}))
unionp


intersect = \hlkwd{function}(c1, c2)\{
	\hlkwd{if} (\hlkwd{len}(c1) == 0)\{
		\hlkwd{c}()
	\}
	else \hlkwd{if} (\hlkwd{is.element}(c1[1], c2))\{
		\hlkwd{append}(\hlkwd{intersect}(c1[-1], c2), c1[1])
	\}
	else\{
		\hlkwd{intersect}(c1[-1], c2)
	\}
\}

intersectp = \hlkwd{intersect}(\hlkwd{c}(\hlstr{"P"},\hlstr{"A"}, \hlstr{"L"}), \hlkwd{c}(\hlstr{"P"},\hlstr{"A"}, \hlstr{"C"}, \hlstr{"N"}))
intersectp

dif = \hlkwd{function}(c1, c2) \{
	res = \hlkwd{c}()
	\hlkwd{for} (element in c1) \{
		\hlkwd{if} (!(element in c2)) \{
			res = \hlkwd{append}(res, element)
		\}
	\}
	res
\}

tabla <- \hlkwd{matrix}(\hlkwd{c}(1,1,0,1,1, 1,1,1,1,0, 1,1,0,1,0, 1,0,1,1,0, 1,1,0,0,0, 0,0,0,1,0),6,5,byrow=TRUE,dimnames=\hlkwd{list}(\hlkwd{c}(\hlstr{"suceso1"},\hlstr{"suceso2"},\hlstr{"suceso3"},\hlstr{"suceso4"},\hlstr{"suceso5"},\hlstr{"suceso6"}),\hlkwd{c}(\hlstr{"P"},\hlstr{"A"},\hlstr{"C"},\hlstr{"L"},\hlstr{"N"})))

tabla

support = \hlkwd{function}(table, elements)\{
	count_support = 0
	\hlkwd{for} (i in 1:\hlkwd{len}(table[,1]))\{
		acum = 1
		\hlkwd{for} (element in elements)\{
			acum = (table[i,element]) & acum
		\}
		count_support = count_support + acum
	\}
	count_support/\hlkwd{len}(table[,1])		
\}

soporte = \hlkwd{support}(tabla, \hlkwd{c}(\hlstr{"P"},\hlstr{"A"}))
soporte


support_clasif = \hlkwd{function}(table, ocurrences, s)\{
	valid_ocurrences = \hlkwd{c}()
	\hlkwd{for} (ocurrence in ocurrences)\{
		support_oc = \hlkwd{support}(table, ocurrence)
		\hlkwd{if} (support_oc >= s)\{
			valid_ocurrences = \hlkwd{append}(valid_ocurrences, ocurrence)
		\}
	\}
	valid_ocurrences	
\}

create_comb = \hlkwd{function}(table, clasif, s) \{
	lista = \hlkwd{c}()
	\hlkwd{for} (i in 2:\hlkwd{len}(clasif)) \{
		comb = \hlkwd{unlist}(\hlkwd{lapply}(i, \hlkwd{function}(m) \{\hlkwd{combn}(clasif, m=m, simplify=TRUE)\}), recursive=FALSE)
		
		\hlkwd{for} (j in \hlkwd{seq}(1, \hlkwd{len}(comb), by=i)) \{
			add = \hlkwd{c}()
			\hlkwd{for} (k in j:(j+i-1)) \{
				add = \hlkwd{append}(add, comb[k])
			\}
			\hlkwd{if} (\hlkwd{support}(table, add) >= s) \{
				lista = \hlkwd{append}(lista, \hlkwd{list}(add))
			\}
		\}
	\}
	lista
\}

confidence = \hlkwd{function}(table, comb) \{
	kMax = \hlkwd{len}(comb[\hlkwd{len}(comb)][[1]])
	
	\hlkwd{for} (i in 2:kMax) \{ \hlcom{#itera dimensiones k}
	
		separar <- \hlkwd{Filter}(\hlkwd{function}(x) \hlkwd{length}(x)==i, comb)
		
		\hlkwd{for} (j in 1:\hlkwd{len}(separar)) \{ \hlcom{#itera elementos de dim k}
			
			\hlkwd{for} (k in 1:i) \{ \hlcom{#itera posibilidades que haya en el lado izquierdo}
			
				cosa = \hlkwd{lapply}(k, \hlkwd{function}(m) \{\hlkwd{combn}(separar[j][[1]], m=m, simplify=TRUE)\})
\hlcom{				#lado = c(cosa[1,], cosa[,1])}
				
				\hlkwd{print}(cosa)
\hlcom{				#print(lado)}
			
			\}
			\hlkwd{print}(\hlstr{"-------"})
		\}
	\}
\}

apriori = \hlkwd{function}(table, s, c) \{
	soporte_clasif = \hlkwd{support_clasif}(tabla, \hlkwd{c}(\hlkwd{c}(\hlstr{"P"}), \hlkwd{c}(\hlstr{"A"}) ,\hlkwd{c}(\hlstr{"L"}), \hlkwd{c}(\hlstr{"C"}), \hlkwd{c}(\hlstr{"N"})), s)
	\hlkwd{print}(soporte_clasif)
	combinaciones = \hlkwd{create_comb}(tabla, soporte_clasif, s)
	\hlkwd{print}(combinaciones)
	conf = \hlkwd{check_conf}()
\}


\hlkwd{apriori}(tabla, 0.5, 0.8)



\end{alltt}


{\ttfamily\noindent\bfseries\color{errorcolor}{\#\# Error: <text>:35:31: inesperado 'in'\\\#\# 34: \ \ \ \ \ \ \ \ for (element in c1) \{\\\#\# 35: \ \ \ \ \ \ \ \ \ \ \ \ \ \ \ \ if (!(element in\\\#\# \ \ \ \ \ \ \ \ \ \ \ \ \ \ \ \ \ \ \ \ \ \ \ \ \ \ \ \ \ \ \ \ \ \ \textasciicircum{}}}\end{kframe}
\end{knitrout}


40 de soporte y 90 de confianza
\end{document}          
